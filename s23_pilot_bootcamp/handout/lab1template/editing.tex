\subsection*{Editing your program}

You can use any editor you wish to write and edit your programs, but
we highly recommend you try out
\lstinline[language={[coin]C}]'VSCode',
\lstinline[language={[coin]C}]'vim', or
\lstinline[language={[coin]C}]'emacs'. These editors are very powerful
and can do much more than just help you edit your code (as you will see).

All three are pre-installed on the GHC cluster computers.

You can also use them from your own computer.  To use
\lstinline[language={[coin]C}]'vim' or
\lstinline[language={[coin]C}]'emacs', you will need to ssh in to one
of the Linux andrew machines; once there, you simply run them
from the terminal prompt as described below.  To use the more modern
\lstinline[language={[coin]C}]'VSCode', you will first need to install
it on your computer by following the few easy steps in the ``VSCode
Guide'' on \gts{} under ``Guides to Success''.


% When ssh'd in to one of the Linux andrew machines,
% \lstinline[language={[coin]C}]'vim' and
% \lstinline[language={[coin]C}]'emacs' can be run directly from the
% terminal without additional setup.  If you prefer working with a more
% modern code editor directly on your own computer, you can set up
% \lstinline[language={[coin]C}]'VSCode' by following a few easy steps
% on the ``VSCode Guide'' on \gts{} under ``Guides to Success''.

For this section of the lab handout, we provide instructions on
working with \lstinline[language={[coin]C}]'VSCode',
\lstinline[language={[coin]C}]'vim' and
\lstinline[language={[coin]C}]'emacs'. We highly recommend that you
\textbf{try all three of them}.  Later, use the one you like
best\footnote{See the
  \href{http://www.cs.cmu.edu/~15122/about.shtml}{course website}
  to learn more about using these editors. They are capable
  of much more than what we describe here.}.

\begin{part}\TAGS{unix}
  Using \underline{\lstinline'cd'}, go into the \lstinline'sample' directory.
  From there open the file \lstinline[language={[coin]C}]'factorial.c0'
  that you downloaded from the previous part of the lab:

  \begin{minipage}[t]{0.4\linewidth}
    \textbf{VIM:}\\ \lstinline[language={[coin]C}]'% vim factorial.c0'

    \medskip
    \textbf{EMACS:}\\ \lstinline[language={[coin]C}]'% emacs factorial.c0'
  \end{minipage}
  % \begin{minipage}[t]{0.3\linewidth}
  %   \textbf{EMACS:}\\ \lstinline[language={[coin]C}]'% emacs factorial.c0'
  % \end{minipage}
  \begin{minipage}[t]{0.5\linewidth}
    \textbf{VSCode:}\\ Start the ``Visual Studio Code'' application and
    open the file \lstinline'factorial.c0' from its ``File'' menu.
  \end{minipage}
\end{part}

You should see the editor start in the Terminal or a \lstinline'VSCode'
window, and you should see a program that looks like it computes
factorial\footnote{For \textbf{Emacs} or \textbf{vim}, trying
  to open a file that doesn't exist (i.e., by making a typo) will create
  a new empty file with that name. If you see a blank file, exit the editor
  and try again.}.
The program is written in C0, the language we'll be using
to start the semester.

\begin{part}\TAGS{unix}
  Edit the program \lstinline[language={[coin]C}]'factorial.c0' and add your
  name and section letter at the appropriate locations. Use the instructions
  below for the editor you're using.

  \textbf{VIM:} This editor has two modes -- \emph{insert mode} for inserting
  text or code, and \emph{command mode} for entering commands. It starts in
  command mode, so you can't edit immediately. Use the arrow keys to
  move around the file. While in command mode, pressing ``\lstinline'i'''
  changes the editor to insert mode, allowing you to type text. Go into
  insert mode and add your name and section letter.
  Press the Escape (ESC) key while in insert mode to return
  to command mode.

  \textbf{EMACS:} You can just start typing and editing without
  hitting special keys. You can use the arrow keys to navigate around
  the file to insert code. There are many shortcuts and built-in
  features to Emacs but you won't need them right now. In the file,
  insert your name and your section letter in the appropriate
  comments in your program.

  \textbf{VSCode:} You can just start typing and use your mouse or
  arrow keys to move around the file. Insert your name and
  section letter in the appropriate comments in your program.
\end{part}

\begin{part}\TAGS{unix}
  Save your changes and exit the editor.

  \textbf{VIM:} Make sure you're in command mode by pressing ESC\@.
  Then, you can save your work and exit vim by entering the
  sequence ``\lstinline':wq''' followed by pressing Enter.
  (If you have unsaved changes you would like to discard,
  you'll have to enter the sequence ``\lstinline':q!''' followed by Enter.
  You can also save without exiting by entering the sequence
  ``\lstinline':w''' followed by Enter.)

  \textbf{EMACS:} Once you're ready to save, press Ctrl-x (the Control
  key and the ``\lstinline'x''' key at the same time) followed by
  Ctrl-s. You can exit by pressing Ctrl-x followed by Ctrl-c.  If you
  have not saved before exiting, Emacs will ask you whether you want
  to save your file (since you changed it) --- press ``\lstinline'y'''
  for yes. (Press ``\lstinline'n''' instead if you don't want to save
  your changes).

  \textbf{VSCode:} To save your work, simply select ``Save'' from the
  ``File'' menu. If you use VSCode regularly, you will want to learn
  some of the shortcuts for common commands, for example Ctrl-s or Cmd-s
  to save a file.

\end{part}

\begin{part}\TAGS{unix}
Try out your new editing skills on the file \lstinline'favorite_number.c0'.
You should see where the file tells you to add a line that returns your
favorite number, like this:
\begin{lstlisting}[language={[coin]C}, belowskip=0pt]
int my_favorite_number() {
    /* add a line below that returns your favorite number */
    return 17; // this is *my* favorite number. Choose your own.
}
\end{lstlisting}
\end{part}
